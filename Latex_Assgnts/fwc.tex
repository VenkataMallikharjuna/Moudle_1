\documentclass[12pt,-letter paper]{article}
\usepackage{siunitx}
\usepackage{setspace}
\usepackage{gensymb}
\usepackage{xcolor}
\usepackage{caption}
%\usepackage{subcaption}
\doublespacing
\singlespacing
\usepackage[none]{hyphenat}
\usepackage{hyperref}
\usepackage{amssymb}
\usepackage{relsize}
\usepackage[cmex10]{amsmath}
\usepackage{mathtools}
\usepackage{amsmath}
\usepackage{commath}
\usepackage{amsthm}
\interdisplaylinepenalty=2500
%\savesymbol{iint}
\usepackage{txfonts}
%\restoresymbol{TXF}{iint}
\usepackage{wasysym}
\usepackage{amsthm}
\usepackage{mathrsfs}
\usepackage{txfonts}
\let\vec\mathbf{}
\usepackage{stfloats}
\usepackage{float}
\usepackage{cite}
\usepackage{cases}
\usepackage{subfig}
%\usepackage{xtab}
\usepackage{longtable}
\usepackage{multirow}
%\usepackage{algorithm}
\usepackage{amssymb}
%\usepackage{algpseudocode}
\usepackage{enumitem}
\usepackage{mathtools}
%\usepackage{eenrc}
%\usepackage[framemethod=tikz]{mdframed}
\usepackage{listings}
%\usepackage{listings}
\usepackage[latin1]{inputenc}
%%\usepackage{color}{   
%%\usepackage{lscape}
\usepackage{textcomp}
\usepackage{titling}
\usepackage{hyperref}
%\usepackage{fulbigskip}   
\usepackage{tikz}
\usepackage{graphicx}
\lstset{
  frame=single,
  breaklines=true
}
\let\vec\mathbf{}
\usepackage{enumitem}
\usepackage{graphicx}
\usepackage{romannum}
\usepackage{siunitx}
\let\vec\mathbf{}
\usepackage{enumitem}
\usepackage{graphicx}
\usepackage{enumitem}
\usepackage{tfrupee}
\usepackage{amsmath}
\usepackage{amssymb}
\usepackage{mwe} % for blindtext and example-image-a in example
\usepackage{wrapfig}
\graphicspath{{figs/}}
\providecommand{\mydet}[1]{\ensuremath{\begin{vmatrix}#1\end{vmatrix}}}
\providecommand{\myvec}[1]{\ensuremath{\begin{bmatrix}#1\end{bmatrix}}}
\providecommand{\cbrak}[1]{\ensuremath{\left\{#1\right\}}}
\providecommand{\brak}[1]{\ensuremath{\left(#1\right)}}

\title{Assignment}
\date{\today}

\begin{document}
\maketitle{ CBSE 2019 Mathematics Questions}
\begin{enumerate}
\item If $A$ is a square matrix of order $3$ with   $\mydet{A}$ = $4$, then write the value of $\mydet {-2 A}$.

\item If
\begin{align*}
    y = \sin ^{-1} {x} + \cos^{-1} {x},
\end{align*}
 find $\dfrac{dy}{dx}$.
 
\item Write the order and the degree of the differential equation
\begin{align*}
    \brak{\dfrac{d^4y}{dx^4}}^2=\brak{x+\brak{\dfrac{dy}{dx}}^2}^3.
\end{align*}

\item If a line has the direction ratios $-18, 12, -4,$ then what are its direction cosines ? 

\item If $\vec{A}=\myvec{0&2\\3&-4}$ and $ k \vec{A}=\myvec{0&3a\\2b&24}$ , then find the values of $k$, $a$ and $b$.

\item Form the differential equation representing the family of curves $y^2=m\brak{a^2-x^2}$ by eliminating the arbitrary constants $'m'$ and $'a'$ .

\item Mother, father and son line up at random for a family photo. If $A$ and $B$ are two events given by $A$ = Son on one end, $B$ = Father in the middle, find $P(B\mid A)$.

\item A coin is tossed $5$ times. Find the probability of getting \begin{enumerate}[label=\roman*.]
    \item  at least $4$ heads
    \item at most $4$ heads.
    \end{enumerate}
    
\item Show that the relation $R$ on the set $Z$ of all integers, given by $R$ =$\cbrak {\brak{a, b}:2 \text{divides} \brak{a - b}}$ is an equivalence relation.

\item If $\tan^{-1}x-\cot^{-1}x =\tan^{-1}\brak{\frac{1}{ \sqrt3}}, x>0$ ,find the value of $x$ and hence find the value of $\sec^{-1}\left(\dfrac{2}{x}\right)$.

\item Using properties of determinants, prove that 
\begin{align*}
    \myvec{b+c&a&a\\b&c+a&b\\c&c&a+b}=4abc.
\end{align*}

\item If $\sin y = x \sin (a + y)$, prove that 
\begin{align*}
    \dfrac{dy}{dx} =\dfrac{\sin^{2}(a+y)}{\sin {a}}
\end{align*}

\item If $(\sin x)^y= x + y$, find $\dfrac{dy}{dx}$.

\item If $y=\brak{\sec^{-1}{x}^{2}},x>0$ , show that 
\begin{align*}
    x^2(x^2-1)\dfrac{d^{2}y}{dx^{2}}+(2x^{3}-x)\dfrac{dy}{dx}-2=0
\end{align*}

\item Find the equations of the tangent and the normal to the curve 
\begin{align*}
y=\dfrac{x-7}{(x-2)(x-3)}    
\end{align*}
at the point where it cuts the x-axis.

\item Find
\begin{align*}
\int\dfrac{\sin{2}x}{(\sin^{2}x+1)(\sin^{2}x+3)} dx
\end{align*}

\item Prove that 
\begin{align*}
\int_{a}^{b} f(x)dx=\int_{a}^{b}{f(a+b-x)dx}
\end{align*}
and hence evaluate
\begin{align*}
    \int_{\pi/6}^{\pi/3}\dfrac{dx}{1+\sqrt{tanx}}.
\end{align*}

\item Solve the differential equations
\begin{align*}
    \dfrac{dy}{dx}=\dfrac{x+y}{x-y}
\end{align*}

\item Solve the differential equations
\begin{align*}
    \brak{1+x}^2dy+2xy dx =\cot{x} \hspace{6pt} dx.
\end{align*} 

\item Find the value of $\lambda$ for which the following lines are perpendicular to each other :
\begin{align*}
    \dfrac{x-5}{5{\lambda+2}}=\dfrac{2-y}{5}=\dfrac{1-z}{-1};\dfrac{x}{1}=\dfrac{y+\dfrac{1}{2}}{2{\lambda}}=\dfrac{z-1}{3}.
\end{align*}
Hence, find whether the lines intersect or not.

\item Find the inverse of the following matrix, using elementary
transformations: 
\begin{align*}
    A=\myvec{2&3&1\\2&4&1\\3&7&2}.
\end{align*}

\item Show that the height of a cylinder, which is open at the top, having a given surface area and greatest volume, is equal to the radius of its base.

\item Find the area of the triangle whose vertices are $\brak{-1, 1},\brak {0, 5}$ and $\brak {3, 2}$,using integration.

\item Find the area of the region bounded by the curves  
\begin{align*}
\brak{x-1}^2 +y^2=1 \\  x^2+y^2=1
\end{align*}
using integration.

\item Find the vector and Cartesian equations of the plane passing through the points $\brak{2,5,-3},\brak{-2,-3,5},$ and $\brak{5,3,-3}$.Also, find the point of intersection of this plane with the line passing through points $\brak{3,1,5} $ and $\brak{-1,-3,-1}.$

\item Find the equation of the plane passing through the intersection of the planes $\overrightarrow r.\brak{\hat{i}+\hat{j}+\hat{k}}=1 $ and$\overrightarrow  r .2\hat{i}+3\hat{j}-\hat{k}+4=0 $ and parallel to $x-axis$. Hence, find the distance of the plane from $x-axis$ .

\item Find the integrating factor of the differential equation 
\begin{align*}
x\dfrac{dy}{dx}-2y  = 2x^{2}
\end{align*}

\item Find $\dfrac{dy}{dx}$,  $\hspace{6pt}  if  \hspace{6pt} $    $xy^{2}-x^{2} = 4$

\item Find the Cartesian equation of the line which passes through the point $\brak{- 2, 4,-5}$ and is parallel to the line 
\begin{align*}
\dfrac{x+3}{3}=\dfrac{4-y}{5}=\dfrac{z+8}{6}.
\end{align*}

\item Let $*$ be a binary operation on $R-\cbrak{-1}$ defined by $a * b = \dfrac{a}{b+1}$ ,for all $a$, $b$ $\in R -\cbrak{-1}$.
Show that $*$ is neither commutative nor associative in $R-\cbrak{-1}$.

\item If $\vec{A} = \myvec{-3&6\\-2&4}$, then show that $A^3= A$.

\item Find
\begin{align*}
    \int \dfrac{\sin x - \cos x}{\sqrt{1+2x}} dx , 0 < x < {\pi/2}.
    \end{align*}
    
    \item Find
    \begin{align*}
\int \dfrac{\sin \brak{x-a}}{\sin\brak{x+a}} dx.
    \end{align*}
    
\item Find 
\begin{align*}
    \int \brak{logx}^2 dx.
\end{align*}

\item Let $X$ be a random variable which assumes values $x1, x2, x3, x4$ such that 
\begin{align*}
2P\brak{X = x1} = 3P\brak{X = x2} = P\brak{X = x3} = 5P\brak{X = x4}.
\end{align*}
Find the probability distribution of $X$.

\item Find a unit vector perpendicular to both the vectors 
$ \overrightarrow a $ and $ \overrightarrow b $ , where $ \overrightarrow a $ = $\hat{i} -7\hat{j} +7\hat{k}$ and $ \overrightarrow b = 3 \hat{i} -2 \hat{j}+ 2 \hat{k}$.

\item Show that the vectors $\hat{i} -2 \hat{j}+ 3 \hat{k},-2 \hat{i}+ 3  \hat{j} -4 \hat{k}$ and $\hat{i} -3 \hat{j}+ 5 \hat{k}$ are coplanar.

\item If $f$ $\brak{x}=\dfrac{4x+3}{6x-4},x \neq  \dfrac{2}{3} $, show that $f \brak{x}= x$ for all $x\neq \dfrac{2}{3}$.Also, find the inverse of $f$.

\item  If
\begin{align*}
 \sin^{-1} \brak{\dfrac{3}{x}} + \sin^{-1}\brak{\dfrac{4}{x}}=\dfrac{\pi}{2} 
\end{align*}
then find the value of $x$.

\item Using properties of determinants, prove that
\begin{align*}
\myvec{a^{2}+1&ab&ac\\ab&b^{2}+1&bc\\ac&bc&c^{2}+1} = 1+a^{2}+b^{2}+c^{2}.
\end{align*}

\item If $y$ = $\brak{\cot^{-1} x}^{2}$, show that
\begin{align*}
\brak{x^2+1}^2 \dfrac{d^2y}{dx^2}+2x\brak{x^{2}+1}\dfrac{dy}{dx}=2.
\end{align*} 

\item Let  $ \overrightarrow a ,\overrightarrow b$ and $\overrightarrow c $ be three vectors such that $\mydet{\overrightarrow{a}}  = 1 ,\mydet{\overrightarrow{b}}= 2$ and $\overrightarrow{|c|} =3$ .If the projection of $\overrightarrow b $ along $\overrightarrow a $ is equal to the projection of $\overrightarrow c $ along $ \overrightarrow a $; and $\overrightarrow b,\overrightarrow c $ are perpendicular to each other, then find $\abs{3\overrightarrow a-2\overrightarrow b +2\overrightarrow c}$.

\item Find the local maxima and local minima, if any, of the following function. Also find the local maximum and the local minimum values, as the case may be :
\begin{align*}
    f\brak{x}=\sin x + \dfrac{1}{2} \cos 2x,0\leq x \leq \dfrac{\pi}{2}.
\end{align*}

\item If $A$ = \myvec{1&-1&1\\2&-1&0\\1&0&0} ,find $A^{2}$ and show that $A^{2} = A^{-1}$ .
\item Using matrix method, solve the following system of equations :
\begin{align*}
    2x-3y+5z&=13\\
    3x+2y-4z &=-2\\
    x+y-2z&=-2.
\end{align*}

\item There are two boxes I and II. Box I contains $3$ red and $6$ black balls. Box II contains $5$ red and $'n'$ black balls. One of the two boxes, box I and box I is selected at random and a ball is drawn at random. The ball drawn is found to be red. If the probability that this red ball comes out from box I is $\dfrac{3}{5}$,find the value of $'n'$.

\item Find the area of the region bounded by the curves $\brak{x-1}^{2} +y^{2} = 1$ and $ x^{2}+y^{2}  = 1$ using integration.

\item $A$ company manufactures two types of novelty souvenirs made of plywood. Souvenirs of type $A$ require $5$ minutes each for cutting and $10$ minutes each for assembling. Souvenirs of type $B$ require $8$ minutes each for cutting and $8$ minutes each for assembling. There are $3$ hours and $20$ minutes available for cutting and $4$ hours available for  assembling. The profit is \rupee $50$ each for type $A$ and \rupee $60$  each for type $B$ souvenirs. How many souvenirs of each type should the company manufacture in order to maximize profit ? Formulate the above LPP and solve it graphically and also find the maximum profit .

\item Find the direction cosines of the line joining the points $P \brak{4, 3, -5}$ and $Q  \brak{-2, 1, -8}$.

\item Find the value of $p$ for which the following lines are perpendicular  :
\begin{align*}
\dfrac{1-x}{3}= \dfrac{2y-14}{2p} = \dfrac{z-3}{2}; \dfrac{1-x}{3p} = \dfrac{y-5}{1} = \dfrac{6-z}{5}.
\end{align*}

\item y=$\sin^{-1}$ + $\cos^{-1}$ , find $\dfrac{dy}{dx}$ .

\item If $A = \myvec{3&9&0\\1&8&-2\\7&5&4}$ and $B = \myvec{4&0&2\\7&1&4\\2&2&6}$ , then find the matrix $B'A'$.

\item Find
\begin{align*}
    \int_{a}^{b}\dfrac{log x}{x}dx.
\end{align*}

\item If $*$ is defined on the set $\textbf{R}$ of all real numbers by :$ a * b =\sqrt{a^{2}+b^{2}}$ ,
find the identity element,if it exists in $\textbf{R}$with respect to $*$.

\item Find the value of $x$, if $\tan \brak{\sec^{-1}\brak{\frac{1}{x}}} = \sin \brak{\tan^{-1}{2}},x > 0$.

\item If $e^{y}\brak{x+1}=1$ , then show that $\dfrac{d^{2}y}{dx^{2}} =\brak{ \dfrac{dy}{dx}}^{2}$ .

\item Find $\dfrac{dy}{dx}$, if $y = \sin ^{-1}\brak{\dfrac{2^{x+1}}{1+4^{x}}}$.

\item Find the intervals in which the function $f$ given by $f(x) = 4x^{3}$ - 6x$^{2}-72x+ 30$ is
\begin{enumerate}
    \item  strictly increasing,
     \item strictly decreasing .
\end{enumerate}

\item Solve the differential equation :
\begin{align*}
    \dfrac{dy}{dx} = \dfrac{x+y}{x-y}.
\end{align*}

\item Solve the differential equation :
\begin{align*}
    \brak{1+x^{2}}dy + 2xy dx =\cot x dx.
\end{align*}

\item Find the area of the region 
\begin{align*}
    \cbrak{(x,y):0\leq y \leq x^{2},0 \leq y \leq x +2 , -1 \leq x \leq 3}.
\end{align*}

\item Evaluate 
\begin{align*}
    \int_{1}^{4}\brak{1+x+e^2x}dx 
\end{align*}
as limit of sums.

\item Find the mean and variance of the random variable $X$ which denotes the number of doublets in four throws of a pair of dice.

\item If \myvec {A} = \myvec{1&1&1\\0&1&3\\1&-2&1},find $A^{-1}$.
Hence, solve the system of equations :
\begin{align*}
    x+y+z&=6,\\
    y+3z&=11\\
    \text  {and} \hspace{12pt}x-2y+z&= 0
\end{align*}

\end {enumerate}
\end{document}
